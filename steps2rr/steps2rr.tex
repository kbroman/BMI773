\documentclass[aspectratio=169,12pt,t]{beamer}
\usepackage{graphicx}
\setbeameroption{hide notes}
\setbeamertemplate{note page}[plain]
\usepackage{listings}

% header.tex: boring LaTeX/Beamer details + macros

% get rid of junk
\usetheme{default}
\beamertemplatenavigationsymbolsempty
\hypersetup{pdfpagemode=UseNone} % don't show bookmarks on initial view


% font
\usepackage{fontspec}
\setsansfont
  [ ExternalLocation = ../fonts/ ,
    UprightFont = *-regular ,
    BoldFont = *-bold ,
    ItalicFont = *-italic ,
    BoldItalicFont = *-bolditalic ]{texgyreheros}
\setbeamerfont{note page}{family*=pplx,size=\footnotesize} % Palatino for notes
% "TeX Gyre Heros can be used as a replacement for Helvetica"
% I've placed them in fonts/; alternatively you can install them
% permanently on your system as follows:
%     Download http://www.gust.org.pl/projects/e-foundry/tex-gyre/heros/qhv2.004otf.zip
%     In Unix, unzip it into ~/.fonts
%     In Mac, unzip it, double-click the .otf files, and install using "FontBook"

% named colors
\definecolor{offwhite}{RGB}{255,250,240}
\definecolor{gray}{RGB}{155,155,155}
\definecolor{purple}{RGB}{177,13,201}
\definecolor{green}{RGB}{46,204,64}

\definecolor{background}{RGB}{255,255,255}
\definecolor{foreground}{RGB}{24,24,24}
\definecolor{title}{RGB}{27,94,134}
\definecolor{subtitle}{RGB}{22,175,124}
\definecolor{hilit}{RGB}{122,0,128}
\definecolor{vhilit}{RGB}{255,0,128}
\definecolor{codehilit}{RGB}{255,0,128}
\definecolor{lolit}{RGB}{95,95,95}
\definecolor{myyellow}{rgb}{1,1,0.7}
\definecolor{nhilit}{RGB}{128,0,128}  % hilit color in notes
\definecolor{nvhilit}{RGB}{255,0,128} % vhilit for notes

\newcommand{\hilit}{\color{hilit}}
\newcommand{\vhilit}{\color{vhilit}}
\newcommand{\nhilit}{\color{nhilit}}
\newcommand{\nvhilit}{\color{nvhilit}}
\newcommand{\lolit}{\color{lolit}}

% use those colors
\setbeamercolor{titlelike}{fg=title}
\setbeamercolor{subtitle}{fg=subtitle}
\setbeamercolor{institute}{fg=lolit}
\setbeamercolor{normal text}{fg=foreground,bg=background}
\setbeamercolor{item}{fg=foreground} % color of bullets
\setbeamercolor{subitem}{fg=lolit}
\setbeamercolor{itemize/enumerate subbody}{fg=lolit}
\setbeamertemplate{itemize subitem}{{\textendash}}
\setbeamerfont{itemize/enumerate subbody}{size=\footnotesize}
\setbeamerfont{itemize/enumerate subitem}{size=\footnotesize}

% page number
\setbeamertemplate{footline}{%
    \raisebox{5pt}{\makebox[\paperwidth]{\hfill\makebox[20pt]{\lolit
          \scriptsize\insertframenumber}}}\hspace*{5pt}}

% add a bit of space at the top of the notes page
\addtobeamertemplate{note page}{\setlength{\parskip}{12pt}}

% default link color
\hypersetup{colorlinks, urlcolor={hilit}}

\lstset{language=bash,
        basicstyle=\ttfamily\scriptsize,
        frame=single,
        commentstyle=,
        backgroundcolor=\color{offwhite},
        showspaces=false,
        showstringspaces=false
        }


% a few macros
\newcommand{\bi}{\begin{itemize}}
\newcommand{\bbi}{\vspace{24pt} \begin{itemize} \itemsep8pt}
\newcommand{\ei}{\end{itemize}}
\newcommand{\be}{\begin{enumerate}}
\newcommand{\bbe}{\vspace{24pt} \begin{enumerate} \itemsep8pt}
\newcommand{\ee}{\end{enumerate}}
\newcommand{\sbi}{\begin{itemize} \fontsize{9pt}{9.5}\selectfont}
\newcommand{\sbe}{\begin{enumerate} \fontsize{9pt}{9.5}\selectfont}
\newcommand{\ig}{\includegraphics}
\newcommand{\subt}[1]{{\footnotesize \color{subtitle} {#1}}}
\newcommand{\ttsm}{\tt \small}
\newcommand{\ttfn}{\tt \footnotesize}
\newcommand{\figh}[2]{\centerline{\includegraphics[height=#2\textheight]{#1}}}
\newcommand{\figw}[2]{\centerline{\includegraphics[width=#2\textwidth]{#1}}}


%%%%%%%%%%%%%%%%%%%%%%%%%%%%%%%%%%%%%%%%%%%%%%%%%%%%%%%%%%%%%%%%%%%%%%
% end of header
%%%%%%%%%%%%%%%%%%%%%%%%%%%%%%%%%%%%%%%%%%%%%%%%%%%%%%%%%%%%%%%%%%%%%%

% title info
\title{Steps toward reproducible research}
\author{\href{https://kbroman.org}{Karl Broman}}
\institute{Biostatistics \& Medical Informatics, UW{\textendash}Madison}
\date{\href{https://kbroman.org}{\tt \scriptsize \color{foreground} kbroman.org}
\\[-4pt]
\href{https://github.com/kbroman}{\tt \scriptsize \color{foreground} github.com/kbroman}
\\[-4pt]
\href{https://twitter.com/kwbroman}{\tt \scriptsize \color{foreground} @kwbroman}
\\[-4pt]
{\scriptsize Slides: \href{https://kbroman.org/BMI773/steps2rr.pdf}{\tt kbroman.org/BMI773/steps2rr.pdf}}
}


\begin{document}

% title slide
{
\setbeamertemplate{footline}{} % no page number here
\frame{
  \titlepage

  \note{This lecture is based on slides for a talk I've given a whole
    bunch of times.

    By ``reproducibly,'' I'm referring to ``computational
    reproducibility,'' by which I mean that the data and code for a
    project are packaged together in a way that they can be handed to
    someone else, who can rerun the code and get the same
    results---the same figures and tables. This is surprisingly hard
    to do, and it's even more difficult in the context of a
    collaboration between two or more data analysts.

    This lecture is an overview of reproducible research and the basic
    principles and tools for ensuring that computational work is
    reproducible. In subsequent lectures, I'll be drilling into
    individual tools/topics in more detail.
}
} }


\begin{frame}[fragile,c]{}

\begin{center}
\begin{minipage}[c]{9.3cm}
\begin{semiverbatim}
\lstset{basicstyle=\normalsize}
\begin{lstlisting}[linewidth=9.3cm]
 Karl -- this is very interesting,
 however you used an old version of
 the data (n=143 rather than n=226).

 I'm really sorry you did all that
 work on the incomplete dataset.

 Bruce
\end{lstlisting}
\end{semiverbatim}
\end{minipage}
\end{center}

\note{I'm an applied statistician; my goal is to help people make
  sense of their data. I have a lot of collaborators, and there's
  nothing I enjoy more than puzzling over their data. So I write a lot of
  reports, describing what I've done and what I've learned.

  This is an email I got from a collaborator,
  in response to an analysis report that I had sent him.
  It's always a bit of a shock to get an email like this: what have I
  done? Why am I working with the wrong data, and where is the right data?

  But what he didn't know is that by this point in my life, I'd
  adopted a reproducible workflow.
  Because I'd set things up carefully, I could just substitute in the
  newer dataset, type a single command (``{\tt make}'') to rerun the
  analyses, and get the revised report.

  This is a reproducibility success story. We all make mistakes, but
  if our projects are reproducible, we can nimbly recover from those
  mistakes.

  There is a second important lesson here: At the start of such
  reports, I always include a paragraph about our shared goals, along
  with some brief data summaries. By doing so, he immediately saw that
  I had an old version of the data. If I hadn't done so, we might
  never have discovered my error.
}
\end{frame}


\begin{frame}[c]{}
\centering
{\Large The results in Table 1 don't seem to \\[12pt]
correspond to those in Figure 2.}

\note{My computational life is not entirely rosy. This is the sort of
  email that will freak me out.}
\end{frame}


\begin{frame}[c]{}
\centerline{\Large Where did we get this data file?}

\note{Record the provenance of all data or metadata files.}
\end{frame}



\begin{frame}[c]{}
\centerline{\Large Why did I omit those samples?}

\note{I may decide to omit a few samples. Will I record {\nhilit why}
  I omitted those particular samples?}
\end{frame}



\begin{frame}[c]{}
\centerline{\Large Which image goes with which experiment?}

\note{For experimental biologists, it can be tricky to keep track of
  the vast set of images and experiments they perform.}
\end{frame}



\begin{frame}[c]{}
\centerline{\Large How did I make that figure?}

\note{Sometimes, in the midst of a bout of exploratory data analysis,
  I'll create some exciting graph and have a heck of a time
  reproducing it afterwards.}
\end{frame}


\begin{frame}[c]{}
\centerline{\Large In what order do I run these scripts?}

\note{Sometimes the process of data file manipulation and data
  cleaning gets spread across a bunch of scripts that need to be
  executed in a particular order. Will I record this information? Is
  it obvious what script does what?}
\end{frame}



\begin{frame}[c]{}
\centerline{\Large ``Your script is now giving an error."}

\note{It was working last week. Well, last month, at least.

How easy is it to go back through that script's history to see when
and why it stopped working?}
\end{frame}



\begin{frame}[c]{}
\centerline{\Large ``The attached is similar to the code we used."}

\note{From an email in response to my request for code used for a
  paper.}
\end{frame}




\begin{frame}[c]{}


\centering
\Large

Reproducible

\bigskip

{\color{lolit} vs.}

\bigskip

\only<1>{Replicable}
\only<2 | handout 0>{{\color{background} p} Correct {\color{background} p}}

\note{Computational work is
  {\color{nhilit} reproducible} if the data and code are organized in
  a way that they can be handed to someone else, who can rerun the
  code and get the same results---the same figures and tables.
  {\color{nhilit} Replicable} is more stringent: can
  someone repeat the experiment and get the same results?

  Reproducibility is a minimal standard. That something is
  reproducible doesn't imply that it is correct. The code may have bugs. The
  methods may be poorly behaved. There could be experimental
  artifacts.

  (But reproducibility is probably associated with correctness.)

  Note that some scientists say replicable for what I call
  reproducible, and vice versa.
}
\end{frame}



\begin{frame}[c]{}

   \centerline{\Large \href{https://kbroman.org/steps2rr}{\tt \color{title}
      kbroman.org/steps2rr}}

\note{It was a long, hard process for me to move from my old standard
  practice to a fully reproducible workflow. In thinking through that
  process, I wrote down my thoughts on the basic steps to take towards
  full reproducibility. This forms the basis of what I'll present here.
}
\end{frame}


\begin{frame}[c]{}

\centering
\large

A little bit reproducible \\
is better than not reproducible.

\vspace{12mm}

A little bit open \\
is better than not open.

\vspace{12mm}

Strive to make each project \\
a bit better organized than the last.

\note{
  While it's good to strive for full reproducibility, it can be difficult
  to achieve. But partially reproducible is better than not-at-all
  reproducible. Similarly, making data and code partially open is
  better than nothing.

  Don't try to change every aspect of your workflow all at once. Focus
  on revising one aspect at a time. When you get to the end of a
  project, you may be dissatisfied with the state of things, but don't
  give up. Try to make each project a bit better organized and
  reproducible than the last.
}

\end{frame}





\begin{frame}[c]{Organize your project}

\begin{center}
\large
\only<1|handout:0>{
File organization and naming \\
are powerful weapons against chaos.
}
\only<2>{
Your closest collaborator is you six months ago, \\
but you don't reply to emails.
}
\only<3|handout:0>{Have sympathy for your future self.}
\end{center}

\hfill
{\lolit
\only<1|handout:0>{{\textendash} \href{https://jennybryan.org}{Jenny Bryan}}
\only<2>{(paraphrasing \href{https://twitter.com/kcranstn/status/370914072511791104}{Mark Holder})}
\only<3|handout:0>{}
}

\note{The first thing to do is to make your project
  understandable to others (or yourself, later, when you try to figure
  out what it was that you did).
}
\end{frame}



\begin{frame}[fragile,c]{Organize your project}

\begin{center}
\begin{minipage}[c]{10.3cm}
\begin{semiverbatim}
\lstset{basicstyle=\normalsize}
\begin{lstlisting}[linewidth=10.3cm]
RawData/              Notes/
DerivedData/          Refs/

Python/               ReadMe.txt
R/                    ToDo.txt
Ruby/                 Makefile

Analysis/
Figures/
\end{lstlisting}
\end{semiverbatim}
\end{minipage}
\end{center}

\note{
  Segregate all the materials for a project in one directory/folder on
  your hard drive.

  There will be a lot of files. Organize them in a meaningful way.

  This is the way I organize a project directory. The key principles
  are to put everything related to a project in a common directory,
  but then to separate data from code and separate raw data from
  processed data.

  Write {\tt ReadMe} files to explain what's what. Make sure they stay
  current.
}

\end{frame}




\begin{frame}[fragile,c]{Chaos}

\begin{center}
\begin{minipage}[c]{11.33cm}
\begin{semiverbatim}
\lstset{basicstyle=\scriptsize}
\begin{lstlisting}[linewidth=11.33cm]
AimeeNullSims/    Deuterium/            Ping/
AimeeResults/     ExtractData4Gary/     Ping2/
AnnotationFiles/  FromAimee/            Ping3/
Brian/            GoldStandard/         Ping4/
Chr6_extrageno/   HumanGWAS/            Play/
Chr6_segdis/      Insulin/              Prdm9/
ChrisPlaisier/    Int2_for_Mark/        RBM_PlasmaUrine_2012-03-08/
Code4Aimee/       Islet_2011-05/        Slco1a6/
CompAnnot/        MappingProbes/        StudyLineupMethods/
CondScans/        MultiProbes/          kidney_chr6.R
D2O_2012-02-14/   NewMap/               pck2_sucla2.R
D2O_cellcycle/    Notes/                penalties.txt
D2Ocorr/          NullSims/             transeQTL4Lude/
Data4Aimee/       NullSims_2009-09-10/
Data4Tram/        PepIns_2012-02-09/
\end{lstlisting}
\end{semiverbatim}
\end{minipage}
\end{center}

\note{
  This is a folder on my hard drive, for the project that led me to
  reassess my life.
}

\end{frame}


\begin{frame}[fragile,c]{Choose good names for things}

\begin{center}
\begin{minipage}[c]{13.7cm}
\begin{semiverbatim}
\lstset{basicstyle=\scriptsize}
\begin{lstlisting}[linewidth=13.7cm]
betw_tissue_corr.R     expr_scatterplot_allprobes.R  gve_similarity_alltissues.R
coatcolor_lod.R        expr_scatterplots_dup.R       gve_similarity.R
colors.R               expr_scatterplots_mix.R       gve_supp.R
cover_fig.R            expr_scatterplots_swap.R      insulin_lod.R
eqtl_counts_10.R       expr_swaps.R                  local_eqtl_locations.R
eqtl_counts.R          func.R                        my_plot_map.R
eve_hist.R             genotype_plates.R             my_plot_scanone.R
eve_scheme.R           gve_hist.R                    sex_vs_X.R
eve_similarity.R       gve_new.R                     xchr_fig.R
eve_similarity_supp.R  gve.R                         xist_and_y.R
expr_corr_dup.R        gve_scheme.R
expr_corr_mix.R        gve_similarity_2ndbest.R
\end{lstlisting}
\end{semiverbatim}
\end{minipage}
\end{center}

\note{
  You'll have a lot of files. In addition to organizing them in
  subfolders, it's important to choose good names for them.

  These names of these files largely explain their contents, but
  they're also left rather disorganized.
}

\end{frame}


\begin{frame}[fragile,c]{Choose good names for things}

\begin{center}
\begin{minipage}[c]{8.3cm}
\begin{semiverbatim}
\lstset{basicstyle=\normalsize}
\begin{lstlisting}[linewidth=8.3cm]
fig1.png        fig5.png
fig10.png       fig6.png
fig2.png        fig7.png
fig3.png        fig8.png
fig4.png        fig9.png
\end{lstlisting}
\end{semiverbatim}
\end{minipage}
\end{center}

\note{
  These names are well organized, but you have to remember the order
  of all of the figures to find the one you want.

  And note that, alphabetically, figure 10 ends up between figure 1
  and figure 2.
}

\end{frame}






\begin{frame}[c]{Choose good names for things}

  \bbi
  \item Machine readable
    \bi
    \item No spaces
    \item No special characters except {\textunderscore} and
      {\tt -}
    \ei

  \item Human readable
    \bi
    \item Explain the contents
    \ei

  \item Consistent
    \bi
      \item Name similar files in a similar way
    \ei

  \item Make use of computer's sorting
    \bi
      \item pad numbers with 0's (e.g., {\tt 01}, {\tt 02}, ...)
      \item start with general grouping, then more specific
      \item dates like {\tt 2019-05-14}
    \ei

  \ei

\note{
  You want the names to be easily to handle in software, which
  generally means no spaces or special characters except for
  underscore and hyphen (which are useful for separating words).

  But you want the names to explain the files' contents, so that you
  don't have to open the files to figure out what they are.

  Consistency is important: if you have a bunch of similar files, you
  should have some system for naming them.

  And make use of the computer's sort of files, by padding numbers
  with 0's (so that 10 appears after 9 rather than before 2) and
  organizing the files into groups.

  Dates should always be written as `YYYY-MM-DD`, so that when sorted
  they are in order by date.
}

\end{frame}


\begin{frame}[c]{}

\vspace{24pt}

\figh{Figs/iso_8601.png}{0.8}

\vfill

\hfill {\tt \footnotesize \lolit \href{http://xkcd.com/1179/}{xkcd.com/1179}}

\note{Go with the xkcd format for writing dates, for ease of sorting.
}
\end{frame}


\begin{frame}[fragile,c]{Choose good names for things}

\begin{center}
\begin{minipage}[c]{10.3cm}
\begin{semiverbatim}
\lstset{basicstyle=\normalsize}
\begin{lstlisting}[linewidth=10.3cm]
0_vcf2db.R
1_prep_geno.R
2_prep_pheno_clin.R
2_prep_pheno_otu.R
3_prep_covar.R
4_prep_analysis_pheno_clin.R
4_prep_analysis_pheno_otu.R
5_scans.R
6_grab_peaks.R
7_find_nearby_peaks.R
\end{lstlisting}
\end{semiverbatim}
\end{minipage}
\end{center}

\note{
  Here's an example to take advantage of the way the computer sorts
  files: a set of R scripts, which show up in the order they are used.
}

\end{frame}



\begin{frame}[c]{No ``{\hilit final}'' in file names}

\vspace*{3mm}

\centering

% comic from http://www.phdcomics.com/comics/archive.php?comicid=1531
\figh{Figs/phd101212s.png}{0.8}

\note{
  Never include ``final'' in a file name.
}

\end{frame}



\begin{frame}<handout:0>[fragile,c]{No ``{\hilit final}'' in file names}


\addtocounter{framenumber}{-1}


\begin{center}
\begin{minipage}[c]{9.5cm}
\begin{semiverbatim}
\lstset{basicstyle=\tiny}
\begin{lstlisting}[escapechar=!,linewidth=9.5cm]
!{\color{foreground}{Deprecated/                            hypo_prcomp.RData}!
!{\color{foreground}{ReadMe.txt                             islet_int1_final.RData}!
!{\color{foreground}{adipose_int1_final.RData               islet_int2_final.RData}!
!{\color{foreground}{adipose_int2_final.RData               islet_mlratio_final.RData}!
!{\color{foreground}{adipose_mlratio_final.RData            islet_mlratio_nqrank_final.RData}!
!{\color{foreground}{adipose_mlratio_nqrank_final.RData     islet_prcomp.RData}!
!{\color{foreground}{adipose_prcomp.RData                   kidney_int1_final.RData}!
!{\color{foreground}{aligned_geno_with_pmap.RData           kidney_int2_final.RData}!
!{\color{foreground}{batches_final.RData                    kidney_mlratio_final.RData}!
!{\color{foreground}{batches_raw_final.RData                kidney_mlratio_nqrank_final.RData}!
!{\color{foreground}{cpl_final.RData                        kidney_prcomp.RData}!
!{\color{foreground}{d2o_final.RData                        lipomics_final_rev2.RData}!
!{\color{foreground}{gastroc_int1_final.RData               liverTG_final.RData}!
!{\color{foreground}{gastroc_int2_final.RData               liver_int1_final.RData}!
!{\color{foreground}{gastroc_mlratio_final.RData            liver_int2_final.RData}!
!{\color{foreground}{gastroc_mlratio_nqrank_final.RData     liver_mlratio_final.RData}!
!{\color{foreground}{gastroc_prcomp.RData                   liver_mlratio_nqrank_final.RData}!
!{\color{foreground}{hypo_int1_final.RData                  liver_prcomp.RData}!
!{\color{foreground}{hypo_int2_final.RData                  mirna_final.RData}!
!{\color{foreground}{hypo_mlratio_final.RData               necropsy_final_rev2.RData}!
!{\color{foreground}{hypo_mlratio_final_old.RData           plasmaurine_final_rev.RData}!
!{\color{foreground}{hypo_mlratio_nqrank_final.RData        pmark.RData}!
!{\color{foreground}{hypo_mlratio_nqrank_final_old.RData    rbm_final.RData}!
!{\color{foreground}{hypo_omit.RData}!
\end{lstlisting}
\end{semiverbatim}
\end{minipage}
\end{center}



\end{frame}


\begin{frame}[fragile,c]{No ``{\hilit final}'' in file names}


\begin{center}
\begin{minipage}[c]{9.5cm}
\begin{semiverbatim}
\lstset{basicstyle=\tiny}
\begin{lstlisting}[escapechar=!,linewidth=9.5cm]
!{\color{foreground}{Deprecated/                            hypo_prcomp.RData}!
!{\color{foreground}{ReadMe.txt                             islet_int1_final.RData}!
!{\color{foreground}{adipose_int1_final.RData               islet_int2_final.RData}!
!{\color{foreground}{adipose_int2_final.RData               islet_mlratio_final.RData}!
!{\color{foreground}{adipose_mlratio_final.RData            islet_mlratio_nqrank_final.RData}!
!{\color{foreground}{adipose_mlratio_nqrank_final.RData     islet_prcomp.RData}!
!{\color{foreground}{adipose_prcomp.RData                   kidney_int1_final.RData}!
!{\color{foreground}{aligned_geno_with_pmap.RData           kidney_int2_final.RData}!
!{\color{foreground}{batches_final.RData                    kidney_mlratio_final.RData}!
!{\color{foreground}{batches_raw_final.RData                kidney_mlratio_nqrank_final.RData}!
!{\color{foreground}{cpl_final.RData                        kidney_prcomp.RData}!
!{\color{foreground}{d2o_final.RData                       }!!{\color{vhilit} lipomics_final_rev2.RData}!
!{\color{foreground}{gastroc_int1_final.RData               liverTG_final.RData}!
!{\color{foreground}{gastroc_int2_final.RData               liver_int1_final.RData}!
!{\color{foreground}{gastroc_mlratio_final.RData            liver_int2_final.RData}!
!{\color{foreground}{gastroc_mlratio_nqrank_final.RData     liver_mlratio_final.RData}!
!{\color{foreground}{gastroc_prcomp.RData                   liver_mlratio_nqrank_final.RData}!
!{\color{foreground}{hypo_int1_final.RData                  liver_prcomp.RData}!
!{\color{foreground}{hypo_int2_final.RData                  mirna_final.RData}!
!{\color{foreground}{hypo_mlratio_final.RData              }!!{\color{vhilit} necropsy_final_rev2.RData}!
!{\color{vhilit}{hypo_mlratio_final_old.RData           plasmaurine_final_rev.RData}!
!{\color{foreground}{hypo_mlratio_nqrank_final.RData        pmark.RData}!
!{\color{vhilit}{hypo_mlratio_nqrank_final_old.RData   }!!{\color{foreground} rbm_final.RData}!
!{\color{foreground}{hypo_omit.RData}!
\end{lstlisting}
\end{semiverbatim}
\end{minipage}
\end{center}

\note{
    This is an actual directory on my computer. If you include
    {\tt final} in a file name, there's a risk that you'll end up with
    {\tt final{\textunderscore}rev},
    {\tt final{\textunderscore}rev2}, and
    {\tt final{\textunderscore}old}.

    Another problem here is that the files aren't organized very well.
  }


\end{frame}


\begin{frame}[fragile,c]{Choose good names for things}


\begin{center}
\begin{minipage}[c]{9.5cm}
\begin{semiverbatim}
\lstset{basicstyle=\tiny}
\begin{lstlisting}[escapechar=!,linewidth=9.5cm]
batches_raw_v1.rds               geneexpr_mlratio_gastroc_v2.rds
batches_v1.rds                   geneexpr_mlratio_hypo_v1.rds
clinical_cpl_v2.rds              geneexpr_mlratio_hypo_v2.rds
clinical_d2o_v2.rds              geneexpr_mlratio_islet_v2.rds
clinical_lipomics_v4.rds         geneexpr_mlratio_kidney_v2.rds
clinical_liverTG_v2.rds          geneexpr_mlratio_liver_v2.rds
clinical_mirna_v2.rds            geneexpr_mlratio_nqrank_adipose_v2.rds
clinical_necropsy_v4.rds         geneexpr_mlratio_nqrank_gastroc_v2.rds
clinical_plasmaurine_v3.rds      geneexpr_mlratio_nqrank_hypo_v1.rds
clinical_rbm_v2.rds              geneexpr_mlratio_nqrank_hypo_v2.rds
Deprecated/                      geneexpr_mlratio_nqrank_islet_v2.rds
geneexpr_int1_adipose_v2.rds     geneexpr_mlratio_nqrank_kidney_v2.rds
geneexpr_int1_gastroc_v2.rds     geneexpr_mlratio_nqrank_liver_v2.rds
geneexpr_int1_hypo_v2.rds        geneexpr_omit_hypo.rds
geneexpr_int1_islet_v2.rds       geneexpr_prcomp_adipose_v2.rds
geneexpr_int1_kidney_v2.rds      geneexpr_prcomp_gastroc_v2.rds
geneexpr_int1_liver_v2.rds       geneexpr_prcomp_hypo_v2.rds
geneexpr_int2_adipose_v2.rds     geneexpr_prcomp_islet_v2.rds
geneexpr_int2_gastroc_v2.rds     geneexpr_prcomp_kidney_v2.rds
geneexpr_int2_hypo_v2.rds        geneexpr_prcomp_liver_v2.rds
geneexpr_int2_islet_v2.rds       geno_aligned_w_pmap.rds
geneexpr_int2_kidney_v2.rds      geno_pmark.rds
geneexpr_int2_liver_v2.rds       ReadMe.txt
geneexpr_mlratio_adipose_v2.rds
\end{lstlisting}
\end{semiverbatim}
\end{minipage}
\end{center}

\note{
    This is the same set of files, renamed. Using {\tt
    clinical{\textunderscore}} and {\tt geneexpr{\textunderscore}}
    brings similar files together.

    A lot of files, but less forbidding.
  }

\end{frame}



\begin{frame}[c]{Document your work}

  \bbi
\item What is all of this stuff?
\item What was your analysis process?
\vspace{1cm}
\item[$\boldsymbol{\rightarrow}$] {\large {\tt ReadMe} files}
  \ei

  \note{
    An overall {\tt ReadMe} file plus an additional such file in each
    directory.

    Well-named files and directories makes everything easier.

    Also, keep the documentation current. There's nothing worse than
    documentation that is out of date and doesn't match the contents.
  }


\end{frame}




\begin{frame}[c]{Organizing data in spreadsheets}


\figw{Figs/bad_spreadsheet.pdf}{1.0}



  \note{
     How you organize your data within files can have a big impact on
     how easy they are to work with.

     You can probably figure out what the numbers mean here,
     particularly if I tell you that there were triplicate
     measurementss under two treatments (1 min or 5 min) of cells that
     were either normal or mutant and cam from mouse strains B6 or
     BTBR.

     But it's hard to tell a computer program about the data structure
     here.
  }
\end{frame}



\begin{frame}[c]{Organizing data in spreadsheets}


\figh{Figs/good_spreadsheet.pdf}{0.8}

  \note{
     This is the first few rows of a reorganized version of the data,
     as a rectangle where the rows are individual measurements and the
     columns are variables.

     This is maybe less pretty, but it's much easier to work with.
  }

\end{frame}




\begin{frame}[c]{Organizing data in spreadsheets}


  \bbi
\item Make it a rectangle
\item Individual measurements as rows; variables as columns
\item Single header row
\item One item per cell
\item No empty cells
\item No calculations in the raw data
\item No highlighting or coloring as data
  \ei

\vspace{8mm}

\hfill
\href{https://doi.org/gdz6cm}{\footnotesize
  \lolit \tt Broman and Woo (2018) Am Stat 72:2-10 \\
  \hfill doi.org/gdz6cm}

  \note{
    Here are some key principles for organizing data in spreadsheets:
    make a rectangle with a single header row.

    Never do calculations in your raw data file. If you're doing
    analyses or making charts in Excel, do so in a copy of the data
    file. Every time you open the raw data file, there's a risk that
    you'll mess things up.
  }
\end{frame}




\begin{frame}[c]{}

\begin{center}
  \Large


``What the heck is `{\hilit \tt FAD{\textunderscore}NAD SI 8.3{\textunderscore}3.3G}'?''

\end{center}

\note{
  Sometimes the columns in your data files have meaning only to you.

  If the data analyst can't connect to the measurements, they're just
  columns of numbers.
}

\end{frame}




\begin{frame}[c]{Metadata}

  \bbi
\item Create a data dictionary
  \bi
\item Explain each column
\item Include different versions of the variable names (compact vs descriptive)
\item Units
\item Allowable values
  \ei
\item The metadata are data
  \bi
\item Make it a rectangle
  \ei
\ei

  \note{
    Clear metadata is critical for others to be able to understand
    your data. In particular, make a data dictionary that describes
    the variables. In addition to a description of each column, I like
    to have short and longer versions of the names for use in data
    visualizations, as the column names themselves can be cryptic.

    These metadata are data, and so rather than make a Word
    documention describing the data, I personally would prefer to have
    another data file with the metadata.
  }

\end{frame}




\begin{frame}[c]{Data dictionary}


\figw{Figs/data_dict.pdf}{1.0}


  \note{
    Here's an example data dictionary. You might also include units
    and informationa about possible valid values.
  }

\end{frame}





\begin{frame}[c]{Everything with a script}

\centering
\large
If you do something once, \\
you'll do it 1000 times.

\note{The most basic principle for reproducible research is: do
  everything via code.

  Downloading data from the web, converting an Excel file to CSV,
  renaming columns/variables, omitting bad samples or data points...do
  all of this with scripts.

  You may be tempted to open up a data file and hand-edit. But if you
  get a revised version of that file, you'll need to do it again. And
  it'll be harder to figure out what it was that you did.

  Some things are more cumbersome via code, but in the long run you'll
  save time.
}
\end{frame}





\begin{frame}[c]{Reproducible reports}


\vspace*{8mm}

\vspace*{-0.05\textheight}
\figw{Figs/example_Rmd.png}{0.92}
\onslide<2|handout 0>{
  \vspace*{-0.70\textheight}
  \hspace*{0.06\textwidth}
  \figw{Figs/example_Rmd_source.png}{0.92}
}

\note{I {\nhilit love} R Markdown for making reproducible reports that
  document the full details of my analysis. R Markdown mixes Markdown
  (for light-weight markup of text) and R code chunks; when processed
  with knitr, the R code is executed and results inserted into the
  final document.

  With these informal reports, I seek to fully capture the entirety of
  my data explorations and decisions.

  Python people should look at Jupyter notebooks.
}
\end{frame}




\begin{frame}[fragile,c]{Automate the process (GNU Make)}

\begin{center}
\begin{minipage}[c]{13.8cm}
\begin{semiverbatim}
\lstset{basicstyle=\footnotesize}
\begin{lstlisting}[escapechar=!,linewidth=13.8cm]
!{\color{foreground}{R/analysis.html}}!: !{\color{foreground}{R/analysis.Rmd Data/cleandata.csv}}!
!{\color{foreground}{    cd R;R -e "rmarkdown::render('analysis.Rmd')"}!

Data/cleandata.csv: R/prepData.R RawData/rawdata.csv
    cd R;R CMD BATCH prepData.R

RawData/rawdata.csv: Python/xls2csv.py RawData/rawdata.xls
    Python/xls2csv.py RawData/rawdata.xls > RawData/rawdata.csv
\end{lstlisting}
\end{semiverbatim}
\end{minipage}
\end{center}

\note{GNU Make is an old (and rather quirky) tool for automating the
  process of building computer programs. But it's useful much more
  broadly, and I find it valuable for automating the full process of
  data file manipulation, data cleaning, and analysis.

  In addition to {\nhilit automating} a complex process, it also
  {\nhilit documents} the process, including the dependencies among
  data files and scripts.
}
\end{frame}



\begin{frame}<handout:0>[fragile,c]{Automate the process (GNU Make)}

\addtocounter{framenumber}{-1}

\begin{center}
\begin{minipage}[c]{13.8cm}
\begin{semiverbatim}
\lstset{basicstyle=\footnotesize}
\begin{lstlisting}[escapechar=!,linewidth=13.8cm]
!{\color{codehilit}{R/analysis.html}}!: !{\color{foreground}{R/analysis.Rmd Data/cleandata.csv}}!
!{\color{foreground}{    cd R;R -e "rmarkdown::render('analysis.Rmd')"}!

Data/cleandata.csv: R/prepData.R RawData/rawdata.csv
    cd R;R CMD BATCH prepData.R

RawData/rawdata.csv: Python/xls2csv.py RawData/rawdata.xls
    Python/xls2csv.py RawData/rawdata.xls > RawData/rawdata.csv
\end{lstlisting}
\end{semiverbatim}
\end{minipage}
\end{center}

\end{frame}



\begin{frame}<handout:0>[fragile,c]{Automate the process (GNU Make)}

\addtocounter{framenumber}{-1}

\begin{center}
\begin{minipage}[c]{13.8cm}
\begin{semiverbatim}
\lstset{basicstyle=\footnotesize}
\begin{lstlisting}[escapechar=!,linewidth=13.8cm]
!{\color{foreground}{R/analysis.html}}!: !{\color{codehilit}{R/analysis.Rmd Data/cleandata.csv}}!
!{\color{foreground}{    cd R;R -e "rmarkdown::render('analysis.Rmd')"}!

Data/cleandata.csv: R/prepData.R RawData/rawdata.csv
    cd R;R CMD BATCH prepData.R

RawData/rawdata.csv: Python/xls2csv.py RawData/rawdata.xls
    Python/xls2csv.py RawData/rawdata.xls > RawData/rawdata.csv
\end{lstlisting}
\end{semiverbatim}
\end{minipage}
\end{center}
\end{frame}



\begin{frame}<handout:0>[fragile,c]{Automate the process (GNU Make)}

\addtocounter{framenumber}{-1}

\begin{center}
\begin{minipage}[c]{13.8cm}
\begin{semiverbatim}
\lstset{basicstyle=\footnotesize}
\begin{lstlisting}[escapechar=!,linewidth=13.8cm]
!{\color{foreground}{R/analysis.html}}!: !{\color{foreground}{R/analysis.Rmd Data/cleandata.csv}}!
!{\color{codehilit}{    cd R;R -e "rmarkdown::render('analysis.Rmd')"}}!

Data/cleandata.csv: R/prepData.R RawData/rawdata.csv
    cd R;R CMD BATCH prepData.R

RawData/rawdata.csv: Python/xls2csv.py RawData/rawdata.xls
    Python/xls2csv.py RawData/rawdata.xls > RawData/rawdata.csv
\end{lstlisting}
\end{semiverbatim}
\end{minipage}
\end{center}
\end{frame}







\begin{frame}[c]{Write modular code}

  \bbi
  \item Modular code is easier to understand, maintain, and reuse.
  \item Turn repeated code into functions
  \item Combine useful functions into a package or module
  \ei

  \note{
    Another important step towards reproducibility is to revise your
    code to make it more clear.

    The single most important step towards clear code is to pull out
    complex or repeated code as a separate function.
    This makes your code easier to read and maintain.

    Next, combine those functions together into a package or module.
    It's surprisingly easy to create an R package (see {\tt
      https://kbroman.org/pkg\_primer}) and it's even easier to make a
    Python module.

    When writing functions, try to write them in a somewhat-general
    way and then pull them out of the project as separate package or
    module, so that you (and/or others) may reuse them for other
    purposes.
}

\end{frame}



\begin{frame}[c]{Keeping track of versions}

\bbi
  \item Google drive / Dropbox / Box

  \item Version numbers in file names

  \item Formal version control (e.g., git/GitHub)
    \bi
  \item Browse changes
  \item Try new things without fear of breaking what works
  \item Jump to the state of the project at any time point
  \item Merge simultaneous changes from multiple people
    \ei
\ei


  \note{
    We all struggle to keep track of versions of things.

    Shared drives (like google drive, dropbox, and box) often keep
    track of past versions, but usually there's a time limit (like
    30 days or a year).

    You can make copies of file with a version number appended to
    the name. You might zip up a directory and include the date in the
    zipped file.

    Formal version control has a number of advantages, including easy
    of browsing the history or jumping to a particular time point.
    The ability to merge simultaneous changes from multiple users is a
    key advantage.

    git can be hard to learn; it's designed for pretty hard-core
    programmers. But there are growing learning resources, and the
    long-term payoff is considerable. For collaborative projects, the
    payoff is immediate.
  }

\end{frame}




\begin{frame}<handout:0>[c]{Version control (git/GitHub)}

\only<1>{\addtocounter{framenumber}{-1}}

\centering

\only<1-2>{\figh{Figs/example_repo}{0.80}}
\onslide<2>{
  \vspace{-0.65\textheight}
  \figh{Figs/example_repo_zoom}{0.55}
}
\end{frame}


\begin{frame}[c]{Version control (git/GitHub)}

\vspace*{3mm}

\centering

\only<1|handout 0>{\figh{Figs/example_history}{0.80}}
\only<2>{\figh{Figs/example_commit}{0.80}}
\only<3|handout 0>{\figh{Figs/example_commit_zoom}{0.80}}

\note{
  git has a steep learning curve, but ultimately I think you'll find
  it really helpful.

  The big selling point is in collaboration: merging changes from
  collaborators, and keep your work synchronized.

  Longer term, there's great value in having the entire history of
  changes to your project. If something stops working, you can go
  back to any point in that history to see when it stopped working and
  why.

  With git, you can also work on new features or analyses without fear
  of breaking the parts that are currently working well.
}
\end{frame}




\begin{frame}[c]{Backups}

\bbi
  \item Multiple places, including off-site

  \item Automatic
\ei

\note{
  I can't emphasize enough the importance of backups. And you must
  have a copy off-site. And if it's not automatic, it won't happen.
}

\end{frame}






\begin{frame}{License your software}

\vspace{60pt}

\centerline{\large Pick a license, any license}

\vspace{18pt}

\hfill
{\textendash} \href{https://blog.codinghorror.com/pick-a-license-any-license/}{Jeff Atwood}

\note{
  If you don't pick a license for your software, no one else can use it.

  So if you want to distribute your code so that others can reproduce
  your analyses, you need to pick a license, any license.

  I choose between the MIT license and the GPL.

  Don't use the Creative Commons licenses for code. But feel free to
  use them for other things.
}
\end{frame}





\begin{frame}[c]{Share your stuff}


  \bbi
\item Code
  \bi
\item GitHub / BitBucket
\item Zenodo (archival, with DOIs)
  \ei

\item Data
  \bi
\item Domain-specific repository {\lolit (e.g., dbGAP)}
\item General repository {\lolit (e.g., github, figshare, zenodo, datadryad)}
\item Institutional repository
  \ei

  \ei

  \note{
    A reproducible workflow is valuable even if you don't intend to
    share your work with others.

    But if do want to share, it's best to place things at a
    third-party site. Ideally one that can be trusted as an archive
    and that provides DOIs.

    Place code at GitHub (or the similar site, BitBucket). The only problem
    is that it can't necessarily be trusted to still be there 5 years from now.
    There's an easy way to have ``releases'' archived at zenodo.org
    automatically, with a DOI. So I recommend that.

    For data, it's probably best to use a domain-specific repository,
    if there is an appropriate one. Otherwise, general repositories
    github, figshare, zenodo, or datadryad. Again, github is not ideal
    because it's not archival and doesn't give DOIs.
  }

\end{frame}






\begin{frame}[c]{Summary}

  \begin{enumerate}
  \item Organize your project
  \item Choose good names for things
  \item Document what's what
  \item Organize data as a rectangle
  \item Metadata is data
  \item Everything with a script
  \item Even better: reproducible reports
  \item Automate the process {\lolit (GNU Make)}
  \item Write modular code {\lolit (functions and packages)}
  \item Use version control {\lolit (git/GitHub)}
  \item License your software
  \item Share your data and code
  \end{enumerate}


\note{
  Summaries are always good.

  Again, don't try to change everything at once.
  Reproducibility can be surprisingly hard and requires a daily
  commitment. And here I'm just thinking about a project with a single
  data analyst. A collaboration with multiple analysts is yet harder.
}
\end{frame}


\begin{frame}[c]{Other considerations}

  \begin{itemize}
    \itemsep12pt
  \item Testing
    \begin{itemize}
    \item[] {\lolit are you getting the right answers?}
    \end{itemize}
  \item Software versions
    \begin{itemize}
    \item[] {\lolit will your stuff work when dependencies change?}
    \end{itemize}
  \item Large-scale computations
    \begin{itemize}
    \item[] {\lolit computation time + dependence on cluster environment}
    \end{itemize}
  \item Collaborations
    \begin{itemize}
    \item[] {\lolit coordinating who does what and where things live}
    \end{itemize}
  \end{itemize}

\note{I've focused on issues for small-scale, single-investigator
  projects, and even with that limited scope, I've not covered
  everything.}

\end{frame}



\begin{frame}[c]{Collaboration}

  \bbi
\item Do more, by working in parallel
\item Do more, through diversity of ideas and skills
\item Reproducible pipelines have immediate advantages
\item Tests of reproducibility
\item Code review
  \ei

  \note{
    Collaboration has a lot of advantages, including for
    reproducibility efforts.

    It can be useful to have a pair of people regularly review each
    other's code, but it can be hard to get your busy friends to pay
    attention to your little project. But if you are working together
    on a project, you can more naturally build in some code review.

    Moreover, you can explicitly test the reproducibility of your
    analyses, by having your collaborator rerun your work, and vice
    versa.
  }

\end{frame}



\begin{frame}[c]{Challenges in collaborations}

  \bbi
{\only<2-|handout 0>{\lolit }
\item Shared vision?
\item Compromise
\item Coordination
\item Communication
\item Sharing code and data
\item Synchronization
}
\onslide<2->{\item Weakest link?}
  \ei

  \note{
    Collaboration also has challenges.

    Do you have a shared vision for the reproducibility of the
    project? You'll no doubt need to make some compromises about how
    things are done: you can't both just do things the way you've
    always done them. Careful coordination and regular communication
    are key.

    And then there are the technical challenges of how to share the
    code and data and make sure your two working projects remain in
    sync.

    In a sense, the reproducibility of a collaborative project is
    dependent on the weakest link. If one collaborator refuses to
    fully participate and share their work, the chain is broken.
}

\end{frame}





\begin{frame}[c]{}


\begin{center}
\Large
  {\color{title} Challenges} \\[24pt]
  {\lolit \large (totally hypothetical)}
\end{center}

\note{
  A collaboration like this will pose many challenges. The following are
  {\nhilit totally hypothetical}. Really.
}

\end{frame}





\begin{frame}[c]{}

\begin{center}
  \Large

  ``Could we meet to talk about the data file structure?'' \\[36pt]
  \onslide<2->{``No.''}
\end{center}

\note{
  Say the first of many sets of data are set up in a way that is
  complicated to handle, both in data entry and for analysis. Will
  your collaborator work with you to refine things?

  Or will every new data file require a day of work, so that it can be
  combined with prior data?
}

\end{frame}






\begin{frame}[c]{}

\begin{center}
  \Large


  ``Wait, these results seem to be based \\
  on the older SNP map.''

\end{center}

\note{
  It can be hard to keep in sync across groups in a multi-site
  project. If a problem is discovered and some aspect of data
  preprocessing needs to be redone, will this get communicated to all
  analysis teams, so that relevant analyses get rerun as needed?
}

\end{frame}





\begin{frame}[c]{}

\begin{center}
  \Large


  ``Could you write the methods section?'' \\[36pt]
  ``But I didn't do the work, \\
  and we don't have the code that was used.''



\end{center}

\note{
  Are all teams sharing their work with each other?
}

\end{frame}





\begin{frame}[c]{}

\begin{center}
  \Large


``My data analyst has taken a job at Google.''

\end{center}

\note{
    What happens if a key data analyst leaves the project?
}

\end{frame}



\begin{frame}[c]{}

\begin{center}
  \Large


``Could you do these analyses? X said they would, but they're not
  responding to my emails.''

\end{center}

\note{
  Everyone has multiple things going on, and sometimes there is need
  for rush analyses, say for a grant submission or conference
  presentation. Is there a shared understanding of who will do what
  when, and how emergencies can be handled?

  The organization of a project often depends on the worst day you
  spent on it. If you need to do a bunch of stuff last-minute, will
  you leave the project directory in a mess, or will you clean up
  after yourself?
}

\end{frame}




\begin{frame}[c]{Shared vision}

  \bbi
\item Publication
\item Code \& data sharing
\item Who will do what
\item Timeline
\item Ongoing sharing of methods, results
  \ei

\note{
  Critical for a successful collaboration is that the collaborators
  have a shared vision for the project. We often maybe think about
  being in agreement on the approach to publication and co-authorship.
  But perhaps more difficult is coming to an agreement on data and
  code sharing (what, where, and when?), on who will do what, on how
  soon it will be done, and on the ongoing sharing, among collaborators,
  of detailed methods and results.
}

\end{frame}



\begin{frame}[c]{Shared workspace}

  \bbi
\item Project structure
\item Data and metadata formats
\item Software environment
\item Automated sync {\lolit (or it won't happen)}
  \ei

\note{
    Also important is the technology or engineering of sharing. Can
    the collaborators agree on the project structure, data and
    metadata formats, and the software environment?

    Some groups may use R and some python. This should not pose a problem.

    A key issue is how to keep the multiple groups' work in sync. It
    is best that this can be done automatically. Experience
    demonstrates that if synchronization approach requires some manual
    steps, they will not be done consistently.
}

\end{frame}




\begin{frame}[c]{Technology for sharing}

  \bbi
\item Data
  \bi
\item figshare
\item dropbox / box / google drive
  \ei
\item Code
  \bi
\item github / bitbucket
  \ei
\item Pipeline / workflow
  \bi
\item make / drake / snakemake / rake
  \ei
\item Full environment
  \bi
\item docker containers
\item \href{https://mybinder.org}{\tt mybinder.org} /
  \href{https://wholetale.org}{\tt wholetale.org}
  \ei
  \ei

\note{
  I must admit to not being totally confident about what advice to
  give, regarding the tools to use for sharing data and code among
  collaborators.

  For sharing data, simple options include posting large files on a
  data repository like figshare, or using cloud drive like dropbox,
  box, or google drive.

  For sharing code, I prefer to use a version control system like git,
  with github, bitbucket, or a locally-managed equivalent.

  For sharing the analysis pipeline or workflow, one can
  incorporate a system like make (or drake, snakemake, or rake) with
  the code.

  The full software environment could be replicated across teams using
  docker containers. Binder and Whole Tale are two systems for making
  this easier.
}

\end{frame}




\begin{frame}[c]{}

\begin{center}
\large
The most important tool is the {\hilit mindset},\\
when starting, that the end product \\
will be reproducible.
\end{center}

\hfill
{\lolit
{\textendash} \href{https://odin.mdacc.tmc.edu/~kabaggerly/}{Keith Baggerly}
}

\note{So true. Desire for reproducibility is step one.
}
\end{frame}



\begin{frame}[c]{}

\Large

Slides: \href{https://kbroman.org/BMI773/steps2rr.pdf}{\tt kbroman.org/BMI773/steps2rr.pdf}

\vspace{10mm}

\href{https://kbroman.org}{\tt kbroman.org}

\vspace{10mm}

\href{https://github.com/kbroman}{\tt github.com/kbroman}

\vspace{10mm}

\href{https://twitter.com/kwbroman}{\tt @kwbroman}


\note{
  Here's where you can find me, as well as the slides for this talk.
}
\end{frame}




\end{document}
