\documentclass[aspectratio=169,12pt,t]{beamer}
\usepackage{graphicx}
\setbeameroption{hide notes}
\setbeamertemplate{note page}[plain]
\usepackage{listings}

\input{../LaTeX/header.tex}

%%%%%%%%%%%%%%%%%%%%%%%%%%%%%%%%%%%%%%%%%%%%%%%%%%%%%%%%%%%%%%%%%%%%%%
% end of header
%%%%%%%%%%%%%%%%%%%%%%%%%%%%%%%%%%%%%%%%%%%%%%%%%%%%%%%%%%%%%%%%%%%%%%

% title info
\title{Exploratory data analysis}
\author{\href{https://kbroman.org}{Karl Broman}}
\institute{Biostatistics \& Medical Informatics, UW{\textendash}Madison}
\date{\href{https://kbroman.org}{\tt \scriptsize \color{foreground} kbroman.org}
\\[-4pt]
\href{https://github.com/kbroman}{\tt \scriptsize \color{foreground} github.com/kbroman}
\\[-4pt]
\href{https://twitter.com/kwbroman}{\tt \scriptsize \color{foreground} @kwbroman}
\\[-4pt]
{\scriptsize Slides: \href{https://kbroman.org/BMI773/eda.pdf}{\tt kbroman.org/BMI773/eda.pdf}}
}


\begin{document}

% title slide
{
\setbeamertemplate{footline}{} % no page number here
\frame{
  \titlepage

\note{
   This lecture concerns exploratory data analysis. Techniques for the
   creative investigation of data, to identify problems and generate
   ideas.
}
} }








\begin{frame}[c]{}

\only<1>{
  \Large
  \color{title}
  \centering
  What is exploratory data analysis?
}

\only<2|handout 0>{
  \bigskip \bigskip

  \figh{Figs/eda_cover.jpg}{0.95}
}

\note{
  What is exploratory data analysis? The term comes from John Tukey.
  For that matter the term ``data analysis'' itself is from Tukey.

  I think he would contrast it with say ``confirmatory'' data
  analysis. Exploratory data analysis is all about creative
  investigation to generate new ideas. Confirmatory data analysis is
  about answering specific questions.
}
\end{frame}








\begin{frame}{What is exploratory data analysis?}

\bigskip \bigskip \bigskip

{\color{title}  Tukey:} \, \, Looking at data to see what it seems to say.

\onslide<2>{
\bigskip \bigskip \bigskip

\hfill \begin{minipage}{0.865\textwidth}
  It is important to understand what you {\hilit can do} \\
  before you learn to measure how {\hilit well} you seem to have
  {\hilit done} it.
\end{minipage}
}

\note{
  Here is what Tukey says in the preface of his book. He defines
  exploratory data analysis as ``looking at data to see what it seems
  to say.''
}
\end{frame}








\begin{frame}{Uses of EDA}

  \bbi
\item Get a sense of things
\item Data diagnostics (quality control)
\item Hoping for an ``a-ha'' moment
\item Following up ``huh'' moments
  \ei

\note{
  What is exploratory data analysis good for?

  Personally, I'm either trying to get a sense of things (as Tukey
  said, figure out what is it that you can do with the data), or I'm
  trying to identify potential problems in the data (data cleaning).

  I'm usually hoping that my explorations will lead some new insight
  that I wouldn't otherwise have achieved. But in practice, I'm
  usually following up on some puzzling aspect of the problem.
}
\end{frame}



\end{document}
